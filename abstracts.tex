\selectlanguage{polish}
\chapter*{Streszczenie}


W pracy inżynierskiej zaimplementowany został algorytm automatycznej kontroli głośności dla sygnałów mowy nagranych macierzą mikrofonową. Do zrealizowania użyto zaawansowanego algorytmu przetwarzania sygnałów jakim jest filtr LCMV. Otrzymany algorytm estymuje kierunki nadchodzenia dzwięku za pomocą algorytmu MUSIC i reguluje głośność pojedynczych mówców przy jednoczesnym tłumieniu zakłócenia jakim jest biały szum. Zaimplemntowane rozwiązanie bazuje na nagraniach z macierzy mikrofonowej. W ramach pracy stworzono również specjalny generator symulujący geometrię macierzy mikrofonej i pomieszczeń, w których rozchodzi się dzwięk. W celu zrealizowania pracy zapoznano się z fachową litaraturą naukową i napisano oprogramowanie w języku Python. Wykonana implementacja została poddana licznym testom. Sprawdzono wpływ czynników takich jak ilość źródeł i ich położenia, poziom szumu a także geometria pokoju i mikrofonu na jakość działania algorytmu. Uzyskane wyniki przeanalizowano i opisano. 

Praca ma następującą kompozycję: w rozdziale \ref{chapter-2} autor przedstawia teorię zagadnienia. W rozdziale \ref{chapter-3} przedstawione zostało rozwiązanie wraz z krótkim opisem użytych metod. W kolejnych rozdziałach znajdują się szczegóły implementacyjne (rozdział \ref{chapter-4}) i testy zaimplementowanego rozwiązania (rodział \ref{chapter-5}).

\newpage

\chapter*{Summary}

The goal of the thesis was to implement automatic gain control algorithm for speech signals recorded with microphone matrix. The project was accomplished using LCMV filter which is complex signal processing concept. The implemented method estimates directions of arrival of sound waves using MUSIC algorithm and adjusts loudness of users. Simultaneously it attenuates white noise. The solution is based on microphone array recordings. Author of this dissertation developed generator which simulates geometry of microphone array and room response. In order to write this thesis author conducted deep research including state-of-the-art scientific literature. The effect of this work is software solution written in Python programming language. The implementation was thoroughly tested. Influence of many factors including number of audio sources, signal to noise ratio and geometry of microphone array and room. Obtained results has been analysed and described. 

Structure of the paper is as follows. In chapter \ref{chapter-2} author has shortly described theory of the problem. In chapter \ref{chapter-3} the solution was presented along with short description of used methods. Chapter \ref{chapter-4} contains details regarding implementation. Finally chapter \ref{chapter-5} presents tests of the system.