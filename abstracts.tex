\selectlanguage{polish}
\chapter*{Streszczenie}


W pracy inżynierskiej zaimplementowany został algorytm automatycznej kontroli głośności dla sygnałów mowy nagranych macierzą mikrofonową. Do zrealizowania użyto zaawansowanego algorytmu przetwarzania sygnałów jakim jest filtr LCMV. Otrzymany algorytm estymuje kierunki nadchodzenia dzwięku za pomocą algorytmu MUSIC i reguluje głośność pojedynczych mówców przy jednoczesnym tłumieniu zakłócenia jakim jest biały szum. Zaimplemntowane rozwiązanie bazuje na nagraniach z macierzy mikrofonowej. W ramach pracy stworzono również specjalny generator symulujący geometrię macierzy mikrofonej i pomieszczeń, w których rozchodzi się dzwięk. W celu zrealizowania pracy zapoznano się z fachową litaraturą naukową i napisano oprogramowanie w języku Python. Wykonana implementacja została poddana licznym testom. Sprawdzono wpływ czynników takich jak ilość źródeł i ich położenia, poziom szumu a także geometria pokoju i mikrofonu na jakość działania algorytmu. Uzyskane wyniki przeanalizowano i opisano. 

W rozdziale \ref{chapter-2} autor przedstawia teorię zagadnienia. W rozdziale \ref{chapter-3} przedstawione zostało rozwiązanie wraz z krótkim opisem użytych metod. W kolejnych rozdziałach znajdują się szczegóły implementacyjne(rozdział \ref{chapter-4}) i testy zaimplementowanego rozwiązania( rodział \ref{chapter-5}).