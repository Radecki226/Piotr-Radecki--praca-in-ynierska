\chapter{Testy}
\label{chapter-5}
\section{Konfiguracja}
Dla celów testów przyjęto pewne sztywne założenia. Jako prędkość dzwięku przyjmuje się $343 \dfrac{m}{s}$ Wszystkie testy są przeprowadzane na sygnałach mowy próbkowanych na częstotliwości 4kHz. STFT i ISTFT jest obliczane przy pomocy okna hanninga \cite{hann} ze współczynnikiem nachodzenia okien(ang. overlap) 50$\%$. Jako okno przestrzenne wspomniane w \ref{alg:gprim} także przyjęto okno hanninga o szerokości $20^{\circ}$. Liczba ramek $N$, z których wyliczana jest macierz PSD jest równa 100.Aktualizacja DOA odbywa się co 10 próbek. Współczynniki $\alpha_{\mathrm{LT}}$ i $\alpha_{G}$ mają wartości 0.9. Jako macierz mikrofonową przyjęto kwadratową strukturę o liczbie mikrofonów $M=16$ o równomiernej odegłości między mikrofonami równej $\delta d = 1cm$. Na potrzeby niektórych eksperymentów odległość między mikrofonami była zwiększana.

W eksperymentach, w których włączony jest pogłos symulowany jest pokój o wymiarach 7m x 5m x 3m. Macierz umieszczona jest w punkcie $[2,2,1.6]$ a mówcy znajdują się w odległości 1.5m od niej. Ilustracja znajduje się na rysunku \ref{fig:room}.

\begin{figure}[h!]
    \centering
    \includegraphics[width=0.4\textwidth]{Images/room.png}
    \caption{Symulowany Pokój}
    \label{fig:room}
\end{figure}

\begin{figure}[h!]
    \centering
    \includegraphics[width=0.4\textwidth]{Images/microphone.png}
    \caption{Macierz Mikrofonowa}
    \label{fig:microphone}
\end{figure}

Jako zakłócenie do usunięcia wybrano biały szum. Jego macierz PSD może być zapisana jako:

W sekcjach poniżej kolejno opisywane są wykonywane testy i doświadczenia.

\section{Charakterystyka kierunkowa filtra LCMV}

Poniżej przedstawiono wykresy charakterystyki kierunkowej dla filtra LCMV o następujących parametrach:

\begin{itemize}
    \item $\theta_{0}=30^{\circ}, \,
    \theta_{1}=120^{\circ}, \,
    \theta_{2}=300^{\circ}$
    \item $G(\theta_{0})=4, \,
    G(\theta_{1})=2, \,
    G(\theta_{2})=1$
\end{itemize}
\noindent dla częstotliwości 100Hz, 500Hz i 2kHz.

\begin{figure}[h!]
    \centering
    \includegraphics[width=\textwidth]{Images/directivity0.02m.png}
    \caption{Charakterystyka kierunkowa dla $\Delta d = 2cm$}
    \label{fig:directivity0.02}
\end{figure}

\noindent Załączony obrazek \ref{fig:directivity0.02} pokazuje, że filtr prawidłowo generuje wymuszenia w określonych kierunkach. Właściwości usuwania tła akustycznego zostaną sprawdzone w kolejnych sekcjach.

\noindent Zgodnie z teorią przedstawioną w \cite{mccowan2001} wraz ze wzrostem stosunku $\dfrac{\Delta d}{\lambda}$, gdzie $\lambda$ to długość fali, rośnie kierunkowość filtru. Prowadzi to do sytuacji, w której maksima są bardzo wąskie i wartość charakterystyki kierunkowej silnie fluktuuje w dziedzinie kątów. Nieznaczny błąd estymacji kierunku nadchodzenia fali może więc prowadzić do fatalnych skutków. Zjawisko pojawiania się dużej liczby minimów i maksimów nazywane jest aliasingiem przestrzennym. Zachodzi dla $\Delta d > \dfrac{\lambda}{2}$. Przy założonej prędkości dzwięku dla częstotliwości $f = 4kHz$ granica aliasingu przestrzennego wystąpi dla $\Delta d \approx 4cm $.Dlatego właśnie zdecydowano się wybrać bezpieczną odległość między mikrofonami $\Delta d = 2cm$.

\noindent Warte pokazania są wykresy takich samych filtrów jeśli mikrofony byłyby oddalone od siebie odpowiednio o $\Delta d = 20cm$ i $\Delta d = 2m$:

\begin{figure}[h!]
    \centering
    \includegraphics[width=\textwidth]{Images/directivity0.2m.png}
    \caption{Charakterystyka kierunkowa dla $\Delta d = 20cm$}
    \label{fig:directivity0.2}
\end{figure}

\begin{figure}[h!]
    \centering
    \includegraphics[width=\textwidth]{Images/directivity2m.png}
    \caption{Charakterystyka kierunkowa dla $\Delta d = 2m$}
    \label{fig:directivity2}
\end{figure}

\newpage

\section{Ewaluacja algorytmu MUSIC}

\noindent W celu sprawdzenia działania zaproponowanego algorytmu MUSIC sprawdzone są różne warianty rozłożenia mówców:

\begin{itemize}
    \item Jeden mówca usytuowany na kącie $\theta_{0} = 78^{\circ}$

    \item Trzech mówców usytuowanych na kątach $\theta_{0} = 78^{\circ}, \, \theta_{1} = 192^{\circ}, \, \theta_{2} = 301^{\circ}$
    
\end{itemize}

\noindent Wszystkie powyższe scenariusze będą powtórzone dla następjących konfiguracji odległości między mikrofonami, wartości stosunku sygnału do szumu i obecności pogłosu:

\begin{itemize}
    \item $\mathrm{SNR}=10\mathrm{dB}$ 
    \item $\mathrm{SNR}=-10\mathrm{dB}$
    \item $\mathrm{SNR}=10\mathrm{dB}, \, $ z pogłosem
\end{itemize}

\noindent Te wartości zostały sprawdzone dla $\Delta d = 2cm$ i $\Delta d = 20cm$ dla porównania. Wygenerowane wykresy przedstawiające znalezione DOA mogą być znalezione w tekście jako załączniki \ref{fig:music_10db_2cm}-\ref{fig:music_10db_20cm_reverb}. Na przedstawionych wykresach po lewej stronie znajduje się efekt działania dla jednego źródła a po prawej dla trzech źródeł.

\begin{figure}[H]
    \centering
    \includegraphics[width=0.85\textwidth]{Images/music_10db.png}
    \caption{$\mathrm{SNR}=10\mathrm{dB}, \, \Delta d = 2cm$}
    \label{fig:music_10db_2cm}
\end{figure}

\begin{figure}[H]
    \centering
    \includegraphics[width=0.85\textwidth]{Images/music_-10db.png}
    \caption{$\mathrm{SNR}=-10\mathrm{dB}, \, \Delta d = 2cm$}
    \label{fig:music_-10db_2cm}
\end{figure}

\begin{figure}[H]
    \centering
    \includegraphics[width=0.85\textwidth]{Images/music_10db_reverb.png}
    \caption{$\mathrm{SNR}=10\mathrm{dB}, \, \Delta d = 2cm$ z pogłosem}
    \label{fig:music_10db_2cm_reverb}
\end{figure}

\begin{figure}[H]
    \centering
    \includegraphics[width=0.85\textwidth]{Images/music_10db_0.2m.png}
    \caption{$\mathrm{SNR}=10\mathrm{dB}, \, \Delta d = 20cm$}
    \label{fig:music_10db_20cm}
\end{figure}

\begin{figure}[H]
    \centering
    \includegraphics[width=0.85\textwidth]{Images/music_-10db_0.2m.png}
    \caption{$\mathrm{SNR}=-10\mathrm{dB}, \, \Delta d = 20cm$}
    \label{fig:music_-10db_20cm}
\end{figure}

\begin{figure}[H]
    \centering
    \includegraphics[width=0.85\textwidth]{Images/music_10db_reverb_0.2m.png}
    \caption{$\mathrm{SNR}=10\mathrm{dB}, \, \Delta d = 20cm$ z pogłosem}
    \label{fig:music_10db_20cm_reverb}
\end{figure}

Wyniki dotyczące wpływu szumu nie zaskakują- algorytm działa mniej dokładnie dla niskiego SNR. To samo dotyczy pogłosu, użyty algorytm nie jest odporny na tego typu zakłócenia. Pojawiające się odbite fale wewnątrzpokoju czynią maksima mniej wybitnymi. Najbardziej interesująca obserwacja zdaniem autora pracy to wpływ odległości między mikrofonami. Większa odległość powoduje większą kierunkowość co przekłada się na dokładniejszą estymację. Dla rozłożenia $\Delta \d = 20cm$ i szumie $\mathrm{SNR}=-10\mathrm{dB}$ udaje się dokładnie wyestomować wszystkie trzy kierunki. Dla tego samego stosunku sygnału do szumu i $\Delta d = 2cm$ algorytm estymuje tylko dwa kierunki i to bardzo niedokładnie. Do dalszych testów wybrano jednak system ze stosunkiem sygnału do szumu na poziomie 10dB a mówcy są dość daleko od siebie.Zbytnia kierunkowość powoduje duże problemy z zastosowaniem filtra LCMV i nawet bardzo dokładna estymacja nie rekompensuje tych strat. .Doświadczenia empiryczne pokazały, że cały system działa lepiej dla $\Delta d = 2cm$.

\newpage

\section{Wizualizacja Działania AGC}

W celu oceny działania kontroli głośności i zrozumienia połączeń pomiędzy funkcjami $\Psi$, $\Psi_{\mathrm(LT)}$, $G'$, $\widehat(G)$. 

\section{Testy całego systemy}

Testy całego systemu sprawdzają kilka różnych aspektów projektu inżynierskiego. Są to:

\begin{itemize}
    \item Stosunek sygnału do szumu(SNR)
    \item Ocena działania systemu dla jednego użytkownika
    \item Ocena działania systemu dla wielu użytkowników
    \item Ewaluacja wpływu poruszania się źródła na działanie algorytmu
    \item Sprawdzenie wpływu pogłosu na jakość przetwarzania
    
\end{itemize}
\subsection{SNR}

Obliczenie SNR dla systemu jest obliczane zarówno dla systemu z pojedynczym użytkownikiem jak i z trzema użytkownikami. Aby obliczyć SNR zakłada się, że $\bm{\mathrm}g = \{1,..,1\}$ dla każdego indeksu czasowego. Oznacza to wyłączenie AGC. Z włączonym AGC następowałyby zmiany lokalnej amplitudy sygnału uniemożliwiając sensowny pomiar SNR. Wszystkie pomiary SNR podaje się w skali logarytmicznej.

\noindent Zastosowano następującą normę pomiaru poprawy SNR:
\begin{equation}
    \label{delta_SNR}
    \Delta \mathrm{SNR} = \mathrm{SNR}_{\mathrm{output}} - \mathrm{SNR}_{\mathrm{input}}
\end{equation}

\noindent Gdzie SNR na wejściu $\mathrm{SNR}_{\mathrm{input}}$ jest znane a mając do dyspozycji niezaszumiony sygnał referencyjny $s_{\mathrm{ref}}(n)$ i sygnał zaszumiony $s_{\mathrm{noisy}}(n)$ można także obliczyć stosunek sygnału do szumu dla $N$ próbek sygnału jako:

\begin{equation}
    \label{SNR}
    \mathrm{SNR} = 10 \log \sum_{n=1}^{N}\left(
    \dfrac
    {s_{\mathrm{ref}}(n)}
    {s_{\mathrm{ref}}(n)-s_{\mathrm{noisy}}(n)} \right)^{2}
\end{equation}
\noindent Taki pomysł obliczania SNR zaczerpnięto luźno z \cite{Virtanen2006}.

\noindent Uzyskano następujące wyniki:
\begin{itemize}
    \item Pojedynczy użytkownik: $\mathrm{SNR} \approx 12dB$ 
    \item Trzech użytkowników: $\mathrm{SNR} \approx 6dB$
\end{itemize}

\noindent Wykresy przedstawiające sygnały przed i po odszumaniu przedstawiono na \ref{fig:snr_boost}

\begin{figure}[h!]
    \centering
    \includegraphics[width=\textwidth]{Images/snr_boost.png}
    \caption{Poprawa SNR}
    \label{fig:snr_boost}
\end{figure}

\subsection{Działanie algorytmu dla pojedynczego użytkownika}

Ocenę algorytmu dla pojedynczego użytkownika przeprowadzono w warunkach gdzie użytkownik jest ustawiony na $\theta_{0}=78^{\circ}$. Nagrany sygnał ma długość 50000 próbek co przekłada się na nieco ponad 6 sekund.
\noindent W celu oceny jakości działania sprawdza się dwa warianty zmian sygnału w czasie:
\begin{itemize}
    \item Dziesięciokrotny wykładniczy wzrost amplitudy w czasie trwania sygnału
    \item Dziesięciokrotny wykładniczy spadek amplitudy w czasie trwania sygnału
\end{itemize}

\noindent W obu przypadkach początkowy SNR wynosi 10dB. Wyniki eksperymentu przedstawiono na wykresach \ref{fig:single_user_increasing} i \ref{fig:single_user_decreasing}

\begin{figure}[h]
    \centering
    \includegraphics[width=\textwidth]{Images/single_user_increasing.png}
    \caption{Wzrost amplitudy sygnału dla pojedynczego użytkownika}
    \label{fig:single_user_increasing}
\end{figure}

\begin{figure}
    \centering
    \includegraphics[width=\textwidth]{Images/single_user_decreasing.png}
    \caption{Spadek amplitudy sygnału dla pojedynczego użytkownika}
    \label{fig:single_user_decreasing}
\end{figure}

\noindent Wyniki wymagają krótkiego komentarza. Widoczne jest, że w obu przypadkach mimo, że amplituda sygnału wejściowego silnie zmienia się w czasie, sygnał wyjściowy utrzymuje względnie stały poziom obwiedni. Szczególnie interesujący jest fakt, że w przypadku słabnącego sygnały wejściowego wzmocnienie sygnału powoduje jego wyjście ponad poziom szumu.

\subsection{Działanie algorytmu dla wielu użytkowników}

W przypadku wielu użytkowników nie jest tak łatwo zaprezentować działanie algorytmu za pomocą wykresu wyjściowego. W celu oceny działania algorytmu konieczne będzie obejrzenie zależności wewnętrzych pomiędzy sygnałami $\Phi$, $\Phi_{\mathrm(LT_SLD)}$, $G'$, $\widehat{G}$. Co więcej zachęca się czytelnika pracy do zapoznania się z próbkami znajdującymi się na stronie...











