\chapter{Podsumowanie}
\label{chapter-6}

Celem pracy inżynierskiej było napisanie algorytmu autmoatycznej kontroli głośności dla macierzy mikrofonowej z jednoczesnym tlumieniem tła akustycznego. Temat udało się zrealizować. Sposób działania algorytmu został zaczerpnięty z literatury, w dużej mierze z \cite{Braun2014}, \cite{Thiergart2013} i  \cite{Schmidt1986}. System napisano w języku Python. Otrzymany algorytm przetwarza nagrania z macierzy mikrofonowej. Działanie algorytmu spełnia postawione założenia - steruje głośnością i tłumi zakłócenie jakim jest biały szum. Zdecydowano się zrealizować żądany system za pomocą filtru LCMV i algorytmu MUSIC.

Algorytm poddano licznym testom. Przetestowano wpływ czynników takich jak poziom szumu, ilość mówców, zmiennośc położenia mówców czy geometrię pokoju i mikrofonu na działanie zrealizowanego rozwiązania. Dokonano analizy otrzymanych wyników i wyciągnięto wnioski, które zostały opisane w rodziale \ref{chapter-5}. Wnioski można podsumować w następujący sposób. System dobrze radzi sobie z minimalizacją SNR i z kontrolą głośności. Działanie dla pojedynczego mówcy jest lepsze niż dla wielu mówców. System działa nawet w przypadku sygnału, którego moc jest poniżej szumu. Należy jednak zwrócić uwagę, że założony model nie jest odporny na pogłos. Zbyt duża wartość tego parametru w praktyce uniemożliwia kontrolę głośności.

W planach autora pracy jest pogłębianie wiedzy na temat AGC i ulepszanie systemu. Plany te zakładają ulepszenie algorytmu estymacji DOA tak aby kierunki nadchodzenia fali były wykrywane dla sygnału składającego się ze źródeł położonych bliżej i dla mniejszej wartości stosunku sygnału do szumu. Ponadto kolejnym naturalnym krokiem rozwoju algorytmu jest wprowadzenie odporności na pogłos. W planach jest także próba szybszej implementacji sprzętowej i uruchomienie algorytmu w czasie rzeczywistym.