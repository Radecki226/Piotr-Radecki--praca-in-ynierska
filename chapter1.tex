\chapter{Wstęp}
\section{Wprowadzenie}

Algorytmy automatycznej kontroli głośności(ang. AGC) są elementem współczesnych systemów przetwarzania sygnałów zarówno akustycznych tak jak w poniższej pracy jak i na przykład radiowych \cite{agc_5g}. Celem stosowania takiego algorytmu jest utrzymanie stałej głośności wszystkich mówców w danym systemie. Konkretnym przykładem zastosowania są wideokonferencje \cite{agc_application}, które z uwagi na sytację pandemiczną na świecie zyskały ogromną popularność. Z uwagi na różne czułości mikrofonów uczestników, różną głośność mówienia i odległość od mikrofonu, uczestnicy mogą odczuwać nieprzyjemne dla ucha fluktuacje głośności. Algorytm AGC redukuje te efekty. Ważne jest aby takie algorytmy skutkowały wysokim poziomem sygnału do szumu(ang. SNR) jak i wysokim stosunkiem sygnału do pozostałych interferencji.

Najprostszym rozwiązaniem jest zastosowanie systemu jednokanałowego \cite{Archibald2008}. W takim systemie możlie jest tylko sterowanie głośnością sygnału wejściowego jako całości, na podstawie chwilowej wartości obwiedni. Uniemożliwia on jednak różny poziom wzmocnienia dla różnych mówców i usunięcie interferencji, jeśli współistnieją one w czasie z głównym sygnałem.

Rozwiązaniem bardziej skomplikowanym, zarówno od strony sprzętowej, programistycznej i czasu obliczeń jest użycie systemu z wieloma mikrofonami. Taki system umożliwia pożądane rozróżnienie między mówcami i eliminację interferencji \cite{Thiergart2013}. W systemie takim możliwa jest estymacja kierunków nadchodzenia fali, a co za tym idzie mocy nadchodzącej z wybranych stron i zastosowanie odpowiednich filtrów, które wzmacniają sygnał z żądanych kierunków.

W tej pracy skupiono się właśnie na drugim z wymienionych rodzajów algorytmów AGC. W pracy zostanie przedstawiona praktyczna implementacja bazująca na nagraniach z macierzy mikrofonowych.

\begin{figure}[h]
    \centering
    \includegraphics[width=\textwidth]{Images/setup.png}
    \caption{Graficzna reprezentacja problemu}
    \label{fig:setup}
\end{figure}

\section{Cel}
Celem pracy inżynierskiej była praktyczna implementacja istniejącego algorytmu AGC z tłumieniem tła akustycznego, wraz z estymacją kierunku nadchodzenia fali. Wybrany algorytm powinien spełniać cechy wymienione w poprzedniej sekcji, tj. pozwalać na regulację głośności nagrania osobno dla różnych mówców i minimalizować zakłócenia. Z tego powodu zdecydowano się na użycie tak zwanego filtra LCMV. Opracowany algorytm w założeniu nie ma być algorytmem czasu rzeczywistego. Powinien wykonywać żądane operacje bazując na nagraniach z macierzy mikrofonowej. Kolejnym celem projektu są testy zaimplementowanego systemu w różnych wariantach i warunkach pracy. W tym celu powstanie generator symulujący różne możliwe konfiguracje.

\section{Zakres wykonanej pracy}
Pierwszym z etapów pracy inżynierskiej było zapoznanie się autora pracy z dostępnymi publikacjami opisującymi zarówno tematykę AGC \cite{Braun2014}, \cite{Archibald2008} jak i generalną tematykę przetwarzania sygnałów dla macierzy czujników \cite{Benesty2008} i wybór metody. Wybrana metoda bazuje w głównej mierze na metodzie \cite{Braun2014}.
Następnie dokonany został wybór środowiska pracy- języka Python wraz z paczką NumPy \cite{numpy}. Szczegółowy opis użytych narzędzie znajduje się w rodziale \ref{chapter-4}. Następnie autor pracy napisał symulator umożliwiający generację sygnałów mikrofonowych. Dalsza część realizacji pracy obejmowała- zaplanowanie praktycznej implementacji systemu, wykonanie tej implementacji i testy przy użyciu symulatora. W testach skupiono się na ewaulacji systemu pod kątem działania dla jednego mówcy i wielu mówców. Sprawdzono odporność na zakłócenia i wpływ geometrii czujnika na działania poszczególnych składowych algorytmu.

