\chapter{Analiza Teoretyczna problemu}
\label{chapter-2}
\section{Model systemu}

Model sygnału jest zaczerpnięty z \cite{Thiergart2013} i \cite{Braun2014}, z tą różnicą, że w tej pracy wszystkie zakłócenia traktujemy razem.
Zakłada się, że używanym czujnikiem jest macierz $M$ mikrofonów dookólnych rozłożonych na płaskiej powierzchnii w pozycjach $\bm{\mathrm{d}}_1...\bm{\mathrm{d}}_M$.

Na czujnik pada $L$ fal płaskich, gdzie $M > L$ ośrodku izotropowym i jednorodnym, w którym obecne są zakłócenia.Każda z fal nadbiega z kąta ${\theta}_1...{\theta}_L$ Model może być zapisany w dziedzinie krótkoczasowej transformacji Fouriera(STFT) jako:
\begin{equation}
    \label{equation:2.1}
    \bm{\mathrm{x}}(k,n)
    =
    [\mathrm{X}(k,n,\bm{\mathrm{d}}_{1})
    ...
    \mathrm{X}(k,n,\bm{\mathrm{d}}_{M})]^{T}
\end{equation}
lub alternatywnie jako:
\begin{equation}
    \label{equation:2.2}
    \bm{\mathrm{x}}(k,n)=
    \sum_{l=1}^{L} \bm{\mathrm{x}}_{l}(k,n)
    + \bm{\mathrm{x}}_{\mathrm{n}}(k,n)
\end{equation}

\noindent gdzie $\bm{\mathrm{x}}_l(k,n)$ to wektor natężeń l-tej fali na poszczególnych mikrofonach a $\bm{\mathrm{x}}_{\mathrm{n}}(k,n)$ to wektor natężenia tła akustycznego na mikrofonach.
Oznacza to, że otrzymany na każdym z mikrofonów sygnał jest mieszaniną sygnału z kilku różnych źródeł i zakłóceń.

\begin{figure}[h]
    \centering
    \includegraphics[width=\textwidth]{Images/model.png}
    \caption{Model systemu}
    \label{fig:model}
\end{figure}

\begin{figure}[h]
    \centering
    \includegraphics[width=\textwidth]{Images/direction.png}
    \caption{Opóźnienie fali na mikrofonie}
    \label{fig:direction}
\end{figure}
\noindent Każdy z wektorów $\bm{\mathrm{x}}_l(k,n)$ może być opisany za pomocą równania:
\begin{equation}
    \label{equation:2.3}
    \bm{\mathrm{x}}_l(k,n)=
    \bm{\mathrm{a}}_l(k,n)X_{l}(k,n,\bm{\mathrm{d}}_{0})
\end{equation}
\noindent gdzie $\bm{\mathrm{a}}_l(k,n)X_{l}(k,n,\bm{\mathrm{d}}_{0})$ jest wartością natężenia l-tej fali w punkcie referencyjnym $\bm{\mathrm{d}}_0$. Często w publikacjach, na przykład w \cite{Braun2014} za punkt referencyjny jest wybierane położenie mikrofonu $\bm{\mathrm{d}}_1$. Autor pracy zdecydował się jednak na bardzej ogólne podejście do problemu.

\noindent Wektor $\bm{\mathrm{a}}_l(k,n)$ będzie później przywoływany jako wektor sterujący. Charakteryzuje on przesunięcia fazowe między mikrofonami i jest zależny od kierunku nadchodzenia fali.


\noindent Jako, że definicja wektora sterującego w dziedzinie częstotlowości może być nieintujcyjna, zostanie tu krótko wyjaśniona natura wektora sterującego.

\noindent Zakłada się, że źródła znajdują się w sferze dalekiej, tj. odległość źródła od mikrofonu jest znacznie większa od długości fali i fizycznych wymiarów mikrofonu. Co więcej zakłada się, że mikrofony są na tyle blisko siebie, że nie występuje pomiędzy nimi tłumienie.

\noindent Załóżmy, że na mikrofon pada sygał nadchodzący z kierunku $\theta_{l}$.
Zapis sygnału w punkcie $\bm{\mathrm{d}}_{0} = [0,0]$ to $\mathrm{f}_{l}(t)$. 
Opóźnienie w czasie tego sygnału na mikrofonie położonym w $\bm{\mathrm{d}}_{\mathrm{m}} = [x_{\mathrm{m}},y_{\mathrm{m}}]$ względem punktu $\bm{\mathrm{d}}_{0}$ wynosić będzie zatem pewne $\mathrm{\tau}$. Różnica dróg między punktami wyniesie $\Delta d$ tak jak na rysunku \ref{fig:direction}. Za pomocą prostej trygonometrii może zostać udodnione, że:
\begin{equation}
    \label{equation:2.4}
    \Delta d = -x_{\mathrm{m}}\cos{\theta_{l}} - y_{\mathrm{m}}\sin{\theta_{l}}
\end{equation}
Następnie przyjmując prędkość rozchodzenia się fali w ośrodku $c$ można zapisać, że:
\begin{equation}
    \label{equation:2.5}
    \tau = \dfrac{\Delta d}{c}
\end{equation}

\noindent Przechodząc z sygnałem do dziedziny częstotliwości i zakładając parę transformat Fouriera $\mathrm{f}_{l}(t)\xleftrightarrow{}\mathrm{X}_{l}(f)$ można zapisać, że:
\begin{equation}
    \label{equation:2.6}
    \mathrm{f}_{l}(t-\tau) \xleftrightarrow{} \mathrm{X}_{l}(f)e^{j 2 \pi f t} =
    \mathrm{a}_{l}(f) \mathrm{X}_{l}(f) 
\end{equation}

\noindent W przypadku wielu mikrofonów:
\begin{equation}
    \label{equation:2.7}
    \bm{\mathrm{x}}_l(f)=
    \bm{\mathrm{a}}_l(f)\mathrm{X}_{l}(f)
\end{equation}


\noindent Taki zapis równania może być przeniesiony do dziedziny STFT. Wówczas otrzymane zostanie równanie \ref{equation:2.3}
\noindent W pracy zakłada się, że kierunek nadchodzenia fali nie jest znany ale znana jest ilość źródeł L w systemie.


\newpage
\section{Postawiony Problem}

W wyżej opisanym modelu mogą zostać zdefiniowane następujące potencjalne problemy:
\begin{itemize}
    \item Sygnał $\bm{\mathrm{x}}_{l_{i}}$ może charakteryzować się amplitudą różną od sygnału $\bm{\mathrm{x}}_{l_{j}}$ co będzie skutkować zagłuszaniem jednego z użytkowników systemu przez innego 
    \item Natężenie sygnału $\bm{\mathrm{x}}_{l_{i}}$ może mocno fluktować w czasie z uwagi na przykład na przybliżanie i oddalanie się mówcy od mikrofonów
    \item W systemie występuje tło akustyczne- nieprzyjemny dla ucha szum gaussowski i pogłos, powodujący, że dzwięk mówcy ma charakterystyczny nieprzyjmny efekt audio
\end{itemize}
